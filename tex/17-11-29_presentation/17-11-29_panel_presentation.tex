\documentclass[11pt]{beamer}


\usetheme{metropolis}
\usepackage{appendixnumberbeamer}

\usepackage{booktabs}
\usepackage[scale=2]{ccicons}

\usepackage{pgfplots}
\usepgfplotslibrary{dateplot}

\usepackage{fontawesome}

\usepackage{xspace}
\newcommand{\themename}{\textbf{\textsc{metropolis}}\xspace}

\usepackage{outlines}



%---------------------- Math -----------------------------------------------
\usepackage{amsthm}
%\usepackage{amsmath} \newenvironment{smatrix}{\begin{pmatrix}}{\end{pmatrix}} %USUAL
\newenvironment{smatrix}{\left(\begin{smallmatrix}}{\end{smallmatrix}\right)} %SMALL
%\usepackage{nccmath} \newenvironment{smatrix}{\left(\begin{mmatrix}}{\end{mmatrix}\right)} %MEDIUM
\usepackage{amsmath}
\usepackage{amsfonts}

%----------------------Bibliography --------------------------------------
\usepackage[longnamesfirst]{natbib}
\def\citeapos#1{\citeauthor{#1}'s (\citeyear{#1})}
\bibliographystyle{ecta}




%%% Edit graphics path  %%%
\graphicspath{{../../}}

\title{How Emmanuel Macron took over the French electorate}
\subtitle{A Panel analysis using the French National Election Study}
\date{November 29th, 2017}
\author{Marcel Schliebs}
\institute{AM Panel Analysis | Prof. Dr. Michael Scharkow}
\titlegraphic{\hfill\includegraphics[height=1.0cm]{results/graphics/zu_logo.png}}

\begin{document}

\maketitle

\newpage

\begin{frame}{The French Context}
From being unknown within most of the French electorate and lacking the support of one of the major French parties, Emmanuel Macron managed to rapidly take over the French presidential democracy, securing landslide victories in the presidential (66\% in the second round) and parliamentary (almost 70\% of seats) elections.
\end{frame}

\begin{frame}{Research Question \& Hypotheses}

\begin{itemize}
	\item<1-|alert@1>{Possible Research questions}
		\only<2-6>{
		\begin{itemize}
			\item<2-6 |alert@2> Were personal candidate considerations or policy orientations decisive?
			\item<3-6 |alert@3> Which were decisive policies with which he convinced voters? 
			\item<4-6 |alert@4> What role did tactical/strategic incentives play?
			\item<5-6 |alert@5> Voters from which ideological positions did EM attract?
			\item<6-6 |alert@6> Did voters with larger ideological gaps to EM switch for him later in the campaign?
		\end{itemize}
		}
	\item<7-|alert@7>{Hypotheses}
			\only<8->{
		\begin{itemize}
			\item<8- |alert@8> H1: A greater ideological distance to EM and his platform leads to a smaller probability of voting for him. (\textit{Downs'ian Rational Voter})
			\item<9- |alert@9> H2: The larger the ideological distance to Macron, the later the switching happend, as polls indicated that there was no reasonable alternative to EM (\textit{Strategic Voting consideration})

		\end{itemize}
	}
\end{itemize}
\end{frame} 

\begin{frame}{Data}

\begin{itemize}
	\item<1-|alert@1>{CEVIPOF-ENEF (\textit{Enquete Electorale Francaise})}
	\only<2-6>{
		\begin{itemize}
			\item<2-6 |alert@2> Longitudinal Panel
			\item<3-6 |alert@3> 16 waves from Nov 2015 to Jul 2017
			\item<4-6 |alert@4> 25000 original observations, thereof 15000 who completed all questionnaires (=> low mortality)
			\item<5-6 |alert@5> Effective Survey weights => Using them, I was able to forecast the first round with less than 0.5 percentage points on average for each of the 6 main candidates
			\item<6-6 |alert@6> thousands of variables, also experiments implemented
		\end{itemize}
	}
	\item<7-|alert@7>{Variables \& Operationalization}
	\only<8->{
		\begin{itemize}
			\item<8- |alert@8> Dep.Var: \textit{Vote for Emmanuel Macron (0|1)}
			\item<9- |alert@9> Ind.Var I: \textit{Ideological Distance to En Marche}
			\item<10- |alert@10> Ind.Var II: \textit{Wave: (measured in months until election?)}
			\item<11- |alert@11> Other Covariates: \textit{the usual Michigan-model control variables}
		\end{itemize}}
\end{itemize}
\end{frame} 

\begin{frame}{Model building}

\begin{itemize}
	\item<1-|alert@1>{Model propositions (\textit{Need some more dirty data cleaning before I start analysis})}
	\only<2-4>{
		\begin{itemize}
			\item<2-4 |alert@2> Base-Mod I:\\ $glmer(macron \sim wave + (1 | id), data, family = binomial())$ [\textit{random intercept for ids}]
			\item<3-4 |alert@3> Mod II:\\ $glmer(macron \sim wave * ideol\_distance + (1 + wave | id), data, family = binomial())$ [\textit{additional random slope for wave}]
			\item<4-4 |alert@4> Mod III?: further include $wave:ideol\_distance|id)$ [\textit{including random intercept and slope for interaction effect}]
		\end{itemize}
	}
	\item<5-|alert@5>{Outlook: Bayesian Model}
	\only<6-7>{
		\begin{itemize}
			\item<6-7 |alert@5> Powerful model for estimating many random effects
			\item<7-8 |alert@6> Implementation in STAN
	\end{itemize}}
\end{itemize}
\end{frame} 



%
\begin{frame}{Discussion \& Feedback}
Thank you for you kind attention!\\
presentation, code,  \only<2-4>{\textbf{but no data yet(sorry!)}}\\
on \faGithub  \href{github.com/schliebs/enef\_panel}{github.com/schliebs/enef\_panel}\\
\only<3-3> {Zum Abschluss noch 1 Meme}
\only<4-4> {Zum Abschluss noch 2 Memes}

\begin{figure}[ht!]
	\begin{center}
		%
		\only<3-4> {\subfigure{%
				\includegraphics[width=0.4\textwidth]{results/images/weddingcake_jpg.jpg}
			}%
		}
		\only<4-4>{
			\subfigure{%
				\includegraphics[width=0.4\textwidth]{results/images/coincidence_jpg.jpg}
			}
		}
		
	\end{center}
\end{figure}5\end{frame}


\end{document}
